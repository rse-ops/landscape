Continuous integration is the effort to continually test and build the codebase
while it is being changed by developers. The rationale being: the earlier the
code is tested, the earlier we catch bugs and new changes can be integrated.
"Continuously" integrating is of course out of reach, but it is commonly
accepted as a metric of how agile the code team is.

Integrating changes continuously requires automated testing, and contributions
of small sizes so that they can be reviewed quickly and remain easy to
understand. Those are means to CI, but can be viewed as goals by themselves.

Reviews form a typical bottleneck to continuous integration: it is the
last manual part of the process. This is why it is key to make reviews as
streamlined as possible. This means in particular: Automated testing should
capture as much relevant information as possible ( code style compliance,
compiling in various contexts, unit-testing and testing with high coverage).
Test results should be displayed intelligibly, the reviewer needs to have
confidence that the report is comprehensive. Reproducing tests and the new
feature should be straightforward.

In the DevOps culture this implies working with container orchestration
services. In the context of HPC, the diversity of systems that codes need to
run on turns into a challenge for CI. It is useful yet insufficient to just
test on a standard Linux x64, Mac, and Windows architecture. Most public CI
services are not designed to interact with a batch system. One solution is to
registering a local server or HPC cluster in the CI service provider, with the
caveat that those cluster come with security requirements that often forbit it.

For this reason, research software development may happen on any of these
exotic systems but then testing is rather limited. Work at national labs using
GitLab \cite{Mendoza_undated-lz,noauthor_undated-fv}, and specifically having a
locally deployed Gitlab server with custom runners on various resources
\cite{noauthor_undated-jd} has been a way to bridge the gap between HPC centers
and web-based CI service providers. Managing this infrastructure requires
managing resources, user accounts, time allocations, and machine access, and
a lot of work has been dedicated to match the security requirements.

Another challenge is that there is no standard workflow language for CI.
After the realization that infrastructure-as-code (using a text based, version
controlled workflow implementation) is more maintainable that UI configured CI,
the consensus was that YAML provided a convenient format to implement CI, but
languages varies between CI providers. Best practices suggest keeping scripts
outside of the YAML implementation to allow for easier translation between
languages.
