\section{Introduction}

Successful development, deployment, and maintenance of research software is central to scientific discovery. In the last decade, the role of Research Software Engineer (RSE) \cite{rse-history} has risen to awareness, and fostered a community of combined researchers and software developers that focus almost exclusively on this task. 

While some RSEs work on research software separate from its application, others are embedded in labs and responsible for data processing, analysis, and otherwise running tasks at scale to produce research outputs. These RSEs, whether they be staff at national labs, academic institutions, or private research institutes, historically have used some form of high performance computing (HPC) to achieve this scale \cite{Wikipedia_contributors2021-kg, xsede-history-of-hpc}. 

This traditional practice has slowly been changing with the availability of cloud computing \cite{Scality2020-of}. As the technological gap between HPC and cloud computing is closing \cite{Guidi2020-ht}, and the cloud can equally meet the needs of research groups \cite{noauthor_2020-mn}, Research Software Engineers are presented with the task of working in both spaces. As they discover best practices and tools, there arises the need to write all of this knowledge down. Synthesizing what we know not only identifies what we know, but also what we don't know and where there are gaps that require attention or work. Arguably, a mature community should have awareness of:

\begin{itemize}
\item What are functional categories of need for the community?
\item What are best practices?
\item What tools are out there and recommended for each use case?
\end{itemize}

Further, there is separation between the developers of research software, and those that deploy it as a workflow or service. This problem isn't new, and in fact we can look to cloud computing for inspiration. Although cloud computing goes back to the 1960s \cite{Foote2017-fi} and the term wasn't coined until 1996 \cite{Scality2020-of}, what we are specifically interested in is DevOps -- a movement that sought to bring together development of software and services ("Dev") with their deployment (operations, or "Ops") starting around 2007 \cite{Atlassian_undated-ka}. Interesting, Research Software Engineering is going through the same challenges, and would benefit from the same kind of movement.

This white-paper introduces the concept of RSE-ops, or the intersection between Research Software Engineering and operations, which for research can mean running workflows or services. We present a first effort at defining relevant functional categories for the community, best practices, and the current landscape of potential areas of growth. We hope this structure can provide a basis for inspiring community and initiative around collaborative and meaningful work.
